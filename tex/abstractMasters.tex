%%%%%%%%%%%%%%%%%%%%%%%%%%%%% Abstract %%%%%%%%%%%%%%%%%%%%

\begin{center}
\Large{\textbf{Abstract}}
\end{center}

\vspace{\myparskip}

We present $K_s$-- and $J$--band photometry of \groj\ during two epochs
of observation, and determine the dereddened and absolute magnitudes
of this star system. We derive a range of spectral types (F0--G2 III--IV) for the secondary star
in \groj, using our $J_{0}-K_{0}$ colour estimate for this soft X--ray transient. We find the absolute
magnitude of \groj\ to be similar to that of another long period soft
X--ray transient. %

\vspace{\myparskip}

The first high signal--to--noise $K$--band spectrum of a black hole X--ray
transient system (\groj) is presented. This is used to show that the quiescent contribution of the accretion disk in \groj\
to the total flux of the system in the $K$--band is negligible. We are
therefore able to measure with certainty the binary inclination from the light curve of this stellar system, and place real bounds on the mass of the primary star in \groj. \groj\ is also shown to have
a similar spectrum to that of an F5--F7 III--IV star. %

\vspace{\myparskip}

The ellipsoidal modulation observed in the $K_s$--band is modelled
using \textit{ELC}, to obtain an inclination of $64\degr$--$70\degr$ and a mass ratio of $2.5$--$6$ for the system. These values concur with past results. The derived primary mass, $M_X =
6.8\pm0.7$\,M\sun, suggests that the compact object in \groj\ is a black
hole. %

\vspace{\myparskip}

An attempt is made to model the system in outburst, taking the
ellipsoidal variability of the secondary star and the eclipse of a bright
accretion disk into account. The resultant fit is poor, a consequence
of the asymmetries of the disk and flickering in the $K_s$--band. The
outburst light curve of \groj\ is shown to display an eclipse of the
accretion disk, consistent with the high inclination of the system. %
