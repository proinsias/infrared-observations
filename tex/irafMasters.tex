%%%%%%%%%%%%%%%%%%%%%%%%%%%%% IRAF %%%%%%%%%%%%%%%%%%%%

\chapter{IRAF}
\label{cha:IRAF}

This appendix gives more specifics of the commands used for reduction
and analysis. The parameters of some of the tasks are listed and
discussed, and a description is given of the \texttt{DAOPHOT}
procedure. %

\vspace{\myparskip}

Further details about the \iraf\ tasks are available from the
online \iraf\ help, accessed using the \texttt{help} task
within \iraf, or from the user manuals for \iraf\ %
\cite{Barnes:1993} %
and \texttt{DAOPHOT} %
\cite{Davis:1994}. %

%%%%%%%%%%%%%%%%%%%%%%%%%%%%% Image Header %%%%%%%%%%%%%%%%%%%%

\section{The Image Header}
\label{cha:IRAF:sec:Imhead}

Each \texttt{IMH} image contains a series of text lines, known as the
\textbf{image header}, %
in which the details of the observation are stored. The text serves as
a digital copy of the observation log and is usually automatically
added to the image at the observatory. %

\vspace{\myparskip}

This information is accessed in \iraf\ by using the
\texttt{imheader} task, e.g., \texttt{imheader} \textit{image.imh}
\textit{longheader=yes}. The \texttt{hselect} command can
alternatively be run to output information selectively from the
header. This enables the user to rapidly identify relevant images
based on various properties of the images. For example, the user can
quickly identify all the images of a particular target star system. %

\vspace{\myparskip}

Several important details usually included in the header are:

\begin{itemize}

\item The file name of the image.

\item The location of the corresponding \texttt{IRAF} pixel file.

\item The type of object was observed, whether it was a star, flatfield or
arc.

\item The name of the star or arc observed.

\item The detector and telescope used.

\item The slit employed for spectroscopic observations.

\item The observatory where the observation was made.

\item The name(s) of the observer(s).

\item The filter band selected.

\item The Universal Date and Time at which the observation of the frame began.

\item The Universal Time at the end of the frame observation.

\item The sidereal time of the observation.

\item The integration time per coadd.

\item The number of coadds.

\item The right ascension and declination of the telescope.

\item The airmass of the observation.

\end{itemize}
Depending on the detector used, other information may also be inserted into
the header. %

%%%%%%%%%%%%%%%%%%%%%%%%%%%%% Photometry %%%%%%%%%%%%%%%%%%%%

\section{Photometry}
\label{cha:IRAF:sec:Photometry}

%%%%%%%%%%%%%%%%%%%%%%%%%%%%% Imexamine %%%%%%%%%%%%%%%%%%%%

\subsection{\texttt{Imexamine}}
\label{cha:IRAF:sec:Photometry:subsec:Imexamine}

The \texttt{imexamine} routine is run to extract information
interactively from an image displayed in the \texttt{DS9} image
viewer. A single image may be examined, or a list of images. We used
\texttt{imexamine} to perform aperture photometry on our images of
\groj\ as follows:%

\begin{itemize}

\item
We displayed an image using \texttt{DS9} and called the
\texttt{imexamine} routine within IRAF. %

\item
Next, we edited the parameters for \texttt{imexamine} while the task was
running using the ``\texttt{:epar}'' cursor command. We activated
logging and set the name of the log (\textit{logfile}). %

\item
We then set the \textit{rimexamine} pset parameters
(``\texttt{:epar r}''). These are used during aperture photometry and
include: %


    \begin{description}

    \item[radius]

    Aperture radius for aperture photometry. %

    \item[magzero]

    User defined zero point for magnitude scale. This value was
    determined from: %
    \begin{eqnarray}
    \label{cha:IRAF:sec:Photometry:subsec:Imexamine:eqn:magzero}
    \mathrm{magzero} = 10 + 2.5 \log{f_{10} I} - k \Chi,
    \end{eqnarray}
    where $f_{10}$ is the counts per second detected from a star of
    magnitude 10, $k$ is the extinction coefficient for the
    observatory, and $I$ and $\Chi$ are the total integration time
    and the airmass for the observation, respectively. Then the
    magnitude ($m$) of the star can be calculated by
    \texttt{imexamine} using: %
    \begin{eqnarray}
    \label{cha:IRAF:sec:Photometry:subsec:Imexamine:eqn:mag}
    m = \mathrm{magzero} - 2.5 \log_{10}{F},
    \end{eqnarray}
    where $F$ is the counts from the star, measured by aperture
    photometry. %

    \end{description}

\item
Using the cursor, we selected a star and performed aperture photometry
on that star (``\texttt{a}''). \texttt{Imexamine} outputted properties
of the star such as position, flux and magnitude to the file
\textit{logfile}. %

\item
This procedure was then repeated for the next image. %

\end{itemize}

%%%%%%%%%%%%%%%%%%%%%%%%%%%%% Image Shifting %%%%%%%%%%%%%%%%%%%%

\subsection{Image Shifting}
\label{cha:IRAF:sec:Photometry:subsec:Imshift}

The required shifting of the images in each grid was done
interactively using two \iraf\ scripts. %

\vspace{\myparskip}

The first script displayed each image in the grid using
\texttt{imexamine}. A reference star which is visible in each frame was
chosen. The coordinates of the star in the first frame was obtained
using \texttt{imexamine}: the image was displayed, the cross--hairs
are placed on the star, and the coordinates were calculated using a
centroid algorithm and written to a log file. This procedure was
repeated for the remaining images. %

\vspace{\myparskip}

The second script used the recorded values to determine the shift for
each image. Each frame was shifted, and a corresponding bad pixel map
created, using commands similar to: \texttt{imshift}
\textit{input=image.imh xshift = 27.06 yshift = 31.84
output=image\_sft.imh}. %

%%%%%%%%%%%%%%%%%%%%%%%%%%%%% DAOPHOT Options %%%%%%%%%%%%%%%%%%%%

\subsection{\texttt{DAOPHOT} Options}
\label{cha:IRAF:sec:Photometry:subsec:DAOPHOTOptions}

A PSF model generally consists of two components:
\begin{inparaenum}[(i)]
\item
an analytic function which models the core of the stars, and
\item
one or more look-up tables
\end{inparaenum}
. The tables list the residuals of the analytic fit to the remaining
annuli of the stars, and are used to improve the fit of the annuli. The
\textit{daopars} pset parameters decide the number of look-up tables
and the type of analytical model selected. %

\begin{description}

\item[function]

The type of analytical model(s) to be employed for the PSF. Possible
models are: Gaussian, Moffat, Lorentzian, or Gaussian core with
Lorentzian wings. A value of \textit{auto} for this parameter causes
\texttt{psf} to determine which model fits the data best (i.e., has
the smallest $\Chi^{2}_{r}$ value). For our analysis, we chose a
Gaussian function. %

\item[varorder]

Determines the number of the look-up table used. If
\textit{varorder} is set to $-1$, no look-up table is
used. This is useful for severely undersampled data. A
\textit{varorder} of 0 results in a constant model, which creates
only one look-up table. A variable model is created by
selecting a \textit{varorder} of 1 or 2. These models construct
three and six look-up tables, respectively. This parameter was
set to 0 for our data. %

\item[fitrad]

% 1999summer.txt %
Determines the area where the analytical fit is applied. This
should generally be set to the full-width at half-maximum (FWHM)
of the selected model stars. We selected a \textit{fitrad} value
of 3.2 pixels initially for the fainter stars, and then 4 for
the bright stars.  %

\item[psfrad]

Sets the area where the empirical fit is applied. The brightest star of
interest was selected. \texttt{Imexamine} was run to determine at
what radius from the the centre of this star the flux from the star
was indistinguishable from the noise. The distance (8 pixels) from
this point to the centre of the star was selected as the value for
\textit{psfrad}. %

\end{description}

%%%%%%%%%%%%%%%%%%%%%%%%%%%%% DAOPHOT Procedure %%%%%%%%%%%%%%%%%%%%

\subsection{\texttt{DAOPHOT} Procedure}
\label{cha:IRAF:sec:Photometry:subsec:DAOPHOT}

The \texttt{DAOPHOT} parameters having been set to their correct
values, a list of the stars of interest and their coordinates was
created, either using \texttt{daofind} or \texttt{imexamine}. The
commands then run to obtain the instrumental magnitudes
of the selected stars were:%

\begin{description}

\item[phot]
Performs automatic aperture photometry on each star. %

\item[pstselect]
Uses the output from \texttt{phot} to select possible PSF model
stars. %

\item[psf]
Allows the user to interactively choose suitable PSF model stars from
the candidates, and to create a PSF model. (We generally selected 3 or
4 model stars from a list of approximately 15.) %

\item[nstar]
Uses the PSF to fit the stars of interest, and check the accuracy of
the model. %

\item[substar]
Subtracts the fitted stars from the original image. This image is then
viewed to look for residuals left after the subtraction. %

\item[allstar]
Determines the magnitude of the selected stars. %

\end{description}

%%%%%%%%%%%%%%%%%%%%%%%%%%%%% Spectroscopy %%%%%%%%%%%%%%%%%%%%

\section{Spectroscopy}
\label{cha:IRAF:sec:Spectroscopy}

%%%%%%%%%%%%%%%%%%%%%%%%%%%%% Apall %%%%%%%%%%%%%%%%%%%%

\subsection{\texttt{Apall}}
\label{cha:IRAF:sec:Spectroscopy:subsec:apall}

This routine was used to extract the spectra from the raw
images. \texttt{Apall} automatically distinguishes the spectrum of the
star from the sky background and the noise, and traces the
spectrum. The following parameters determine the extraction process: %

\begin{description}

\item[format]
The format for the extracted spectra. The \textit{multispec} format
was chosen for our spectra. %

\item[references]
A list of images to select as aperture references. This parameter was set
to the null string for the target spectra. For extracting an arc
spectrum, it was set to the corresponding reference image name. %

\item[interactive]
Decides whether the task is run interactively or not. The target
spectra were extracted interactively (\textit{interactive} =
\textit{yes}), whereas the arc spectra were automatically extracted
(\textit{no}). %

\item[background]
Determines what type of background to subtract. No background was
subtracted from our spectra using \texttt{apall} (\textit{background} = \textit{none}), as we subtracted the background manually. %

\item[saturation]
The saturation level of the detector. %

\item[readnoise]
The read out noise of the detector -- this parameter was automatically
read from the \textit{RDNOISE} field of the image headers. %

\item[gain]
The photon gain of the detector -- this was read by \texttt{IRAF} from
the \textit{GAIN} header field. %

\end{description}


%%%%%%%%%%%%%%%%%%%%%%%%%%%%% Identify %%%%%%%%%%%%%%%%%%%%

\subsection{\texttt{Identify}}
\label{cha:IRAF:sec:Spectroscopy:subsec:identify}

\texttt{Identify} was used to display an arc spectrum and match the
spectral features with their corresponding wavelengths. It then
computed a pixel to wavelength conversion function. The parameters modified from their default values for our spectra were: %

\begin{description}

\item[coordlist]
% j1655spectralog.doc pg.5 %
Selects the coordinate list used to identify the arc lines. For our
arcs, we chose the supplied \textit{linelists\$argon.dat}. %

\item[units]
The type of units to be used for the coordinates entered. Our data were
given in \textit{microns}. %

\item[fwidth]
The full-width at zero of the features in the arc spectrum. We determined
this value to be 8 pixels. %

\item[function]
The function to be applied to fit the dispersion correction. We selected a
cubic spline -- \textit{spline3}. %

\item[order]
The order of \textit{function}. Our spline had order 3. %

\end{description}

%%%%%%%%%%%%%%%%%%%%%%%%%%%%% Reidentify %%%%%%%%%%%%%%%%%%%%

\subsection{\texttt{Reidentify}}
\label{cha:IRAF:sec:Spectroscopy:subsec:reidentify}

Rather than using \texttt{identify} on each individual arc, the
\texttt{reidentify} task was run in order to apply the previous
identification of the spectral features in the original arc to
identify the features in successive arcs. The identification process
is modified by changing the following parameters: %

\begin{description}

\item[images]
A list of the images to be reidentified. %

\item[interactive]
Decides whether the task is run interactively or not. %

\item[shift]
The shift to apply to the user-specified coordinates for the arc
features. This was set to \textit{INDEF}, allowing \iraf\ to calculate
any necessary shift. %

\item[search]
When \textit{shift} is set to \textit{INDEF}, this value specifies how
\iraf\ calculates the shift. This was set to \textit{INDEF}. \iraf\ therefore compared the line peaks in the spectra to determine the shift. %

\item[coordlist]
Again, the coordinate list \textit{linelists\$argon.dat} was selected. %

\end{description}

%%%%%%%%%%%%%%%%%%%%%%%%%%%%% Splot %%%%%%%%%%%%%%%%%%%%

\subsection{\texttt{Splot}}
\label{cha:IRAF:sec:Spectroscopy:subsec:splot}

\texttt{Splot} can be used to extract various properties
interactively from a spectrum. Either a single spectrum, or a
group of spectra can be examined. \texttt{Splot} was used in the
following ways with our spectra: %

\begin{itemize}

\item
Having obtained a combined spectrum of our target \groj, we smoothed
this spectrum using the ``\texttt{s}'' cursor
command. This implemented a boxcar smooth. %

\item
We measured the equivalent width of a spectral feature as
follows. First, we viewed the spectrum using \texttt{splot} and
expanded the view of the spectrum to the area around the spectral
line (``\texttt{a}'' command). We then indicated the position of the
feature using the ``\texttt{e}'' command, which then displayed the
wavelength of the feature and the equivalent width. We loaded the next
spectrum by typing ``\texttt{g}'', and repeated the procedure. %

\item
The masking of a spectral feature was performed using the
``\texttt{j}'' command. The view of the spectrum was zoomed into the
spectral feature, and the flux value was set to the continuum
value. This new spectrum was then saved by typing ``\texttt{i}''.

\end{itemize}

%%%%%%%%%%%%%%%%%%%%%%%%%%%%% End of Chapter %%%%%%%%%%%%%%%%%%%%
