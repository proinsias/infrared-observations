%%%%%%%%%%%%%%%%%%%%%%%%%%%%% GRO J1655-40 %%%%%%%%%%%%%%%%%%%%
\chapter{\groj}\label{cha:GROJ1655-40}

In this chapter, we first detail the system properties of \groj, with
emphasis on the accretion disk surrounding the primary star. We list
the observations we made of this SXT, and also elucidate the general reasons for studying this
system. %

\vspace{\myparskip}

Our specific reason for studying \groj\ is to obtain an estimate of
the mass of the black hole which this system is thought to harbour. If
we assume that the accretion disk contributes little to the total flux
from the system, we can then constrain the mass ratio $q$ (and hence the primary mass $M_{X}$) and the inclination $i$  from the ellipsoidal modulation of the binary and the radial velocity curve. %

\vspace{\myparskip}

We will therefore later explain how we applied infrared photometric techniques to obtain the light curve of \groj\ during the period of our observations, and show how we determined the mass of the black hole from the light curve, before discussing whether our assumption of negligible disk
contamination is valid.

%%%%%%%%%%%%%%%%%%%%%%%%%%%%% Introduction to GRO J1655-40 %%%%%%%%%%%%%%%%%%%%

\section{Introduction to \groj}\label{cha:GROJ1655-40:sec:IntroductionToJ1655}

In July 1994, \mbox{GRO J1655--40}\footnote{\label{cha:GROJ1655-40:sec:IntroductionToJ1655:foot:SIMBAD}
The SIMBAD identifier of \groj\ is V* V1033 Sco. It is also
referred to as \textbf{\nova\ (\mbox{Nova Sco 1994})}.%
}%
\ was observed for the first time by the Burst and
Transient Source Experiment (BATSE) onboard the Compton Gamma Ray
Observatory (CGRO) (Zhang et~al.\ \citeyearNP{Zhang_et_al.:1994}). %
Four outbursts in total were observed in the first five months after
its discovery (Harmon et~al.\ \citeyearNP{Harmon_et_al.:1995}).%
\ The optical counterpart of this system was identified in 1995 by
Bailyn et~al.\ \citeyear{BailynOroszGirad_et_al.:1995}%
, the mass function derived by several observers (e.g.,
\citeNP{OroszBailyn:1997}) for the primary star in \groj\ indicated
that this system might contain a black hole. %

%%%%%%%%%%%%%%%%%%%%%%%%%%%%% Properties of J1655 %%%%%%%%%%%%%%%%%%%%

\subsection{General Properties and Special Characteristics}\label{cha:GROJ1655-40:sec:IntroductionToJ1655:subsec:PropertiesOfJ1655}

%%%%%%%%%%%%%%%%%%%%%%%%%%%%% System Properties %%%%%%%%%%%%%%%%%%%%

\subsubsection{System Properties}\label{cha:GROJ1655-40:sec:IntroductionToJ1655:subsec:PropertiesOfJ1655:subsubsec:SystemProperties}

\groj\ lies approximately 3.2 kiloparsecs (kpc) away %
\cite{HjellmingRupen:1995}. Hynes et~al.\ %
\citeyear{Hynes_et_al.:1998}%
\ derived a reddening $E_{B-V}$ of $1.2\pm0.1$\,mag for \groj, from which
they calculate a hydrogen column density of $N_H
\sim5\times10^{21}\,\mathrm{cm}^{-2}$. From these values, the
quiescent bolometric luminosity of \groj\ can be determined to be
approximately $41\,\mathrm{L}\sun$ (see, e.g., %
van~der~Hooft et~al.\ %
\citeyearNP{VanDerHooft_et_al.:1998}).%
\ The quiescent $V$--band magnitude of \groj\ is approximately 17.1
mag %
\cite{GreeneBailynOrosz:2001}, %
and the system becomes as bright as 16.2 mag in the $V$--band during
outburst %
\cite{BailynOroszMcClintockRemillard:1995}. %

\vspace{\myparskip}

The times of the inferior conjunction of the primary are described by the
following photometric ephemeris (van~der~Hooft et~al.\ %
\citeyearNP{VanDerHooft_et_al.:1998})%
, given in MJD%
\footnote{\label{cha:GROJ1655-40:sec:IntroductionToJ1655:subsec:PropertiesOfJ1655:subsubsec:SystemProperties:foot:mjd}%
The \textbf{Modified Julian Date (MJD)} is defined to be the Julian
Date (JD) minus 2400000.5.} %
:
\begin{eqnarray}\label{cha:GROJ1655-40:sec:IntroductionToJ1655:subsec:PropertiesOfJ1655:subsubsec:SystemProperties:eqn:Ephemeris}
\nonumber T_{\mathrm{min}}\ (\mathrm{MJD}) & = &T_0 + P \times N, \\
 & = & 49838.4198(52) + 2.62168(14) \times N,
\end{eqnarray}
where $T_0$ is the time of mid--eclipse, $P$ is the orbital period and $N$
denotes the number of orbital cycles. %

\vspace{\myparskip}

The most recent values for the mass ratio and
inclination of \groj\ are those predicted by %
\citeN{BeerPodsiadlowski:2001}. They derived values of $q=3.9\pm0.6$ and $i=68 \fdg 65 \pm
1 \fdg 5$. Using the mass function of \citeN{Shahbaz_et_al.:1999}, %
$f(M_X) = 2.73\pm0.09\,M\sun$, they determined that
\groj\ consists of a $5.4\pm0.3$\,M\sun\ black hole primary and a
$1.45\pm0.35$\,M\sun\ secondary. The secondary is a giant (or subgiant) of spectra type F5--G0%
, and orbits at a distance of approximately 16.6\,R\sun\ from the primary. %

%%%%%%%%%%%%%%%%%%%%%%%%%%%%% Eccentricities %%%%%%%%%%%%%%%%%%%%

\subsubsection{Anomalies}\label{cha:GROJ1655-40:sec:IntroductionToJ1655:subsec:PropertiesOfJ1655:subsubsec:Eccentricities}

\groj\ displays several atypical characteristics:

\begin{itemize}

\item %
\groj\ is optically the brightest in quiescence of all SXTs. %
The secondary dominates in quiescence to an unusually large extent,
due its relatively luminous F/G star nature.
The light curves of \groj\ are therefore unusually smooth and
symmetrical when compared to those of other black hole binaries, and
so the measurements of the mass ratio and inclination of this system
should therefore be atypically precise. %

\item
\groj\ is one of three known Galactic X--ray sources with apparently superluminal
jets -- another being GRS 1915+105 -- (see Tavani et~al.\ %
\citeyearNP{Tavani_et_al.:1996} %
and references therein), and is the only such source to be identified
optically. \groj\ is also less obscured by dust than \mbox{GRS 1915+105}, and is hence more easily studied.
The jets were discovered by Tingay et~al.\ %
\citeyear{Tingay_et_al.:1995} %
and were shown to have actual speeds of $0.92 \pm 0.02\,c$ by %
\citeN{HjellmingRupen:1995}. %

Most other SXTs have shorter periods and smaller accretion disks %
(Hynes et~al.\ %
\citeyearNP{Hynes_et_al.:1998}).%

\item Kolb et~al.\ %
\citeyear{Kolb_et_al.:1997}%
\ suggested that \groj\ is in a different evolutionary state to other
SXTs. %
Although the secondary stars in SXTs are evolved, the secondary in
\groj\ is an extreme case, and has almost become a giant. %

\item
\groj\ has an unusually high accretion rate ($\dot{M}_{2}$), which
is close to its critical value ($\dot{M}_{\mathrm{crit}}$): these
rates were calculated by %
\citeN{OroszBailyn:1997}%
\ to be $3.4\times10^{-9}\,\mathrm{M}\sun\,\mathrm{yr}^{-1}$ and
$1.1\times10^{-8}\,\mathrm{M}\sun\,\mathrm{yr}^{-1}$,
respectively. This implies that \groj\ is likely to have frequent
X--ray outbursts. %



\item \groj\ is the only SXT to show evidence of eclipses, as observed by
Bailyn et al.\ %
\citeyear{BailynOroszGirad_et_al.:1995}%
\ at optical wavelengths. These eclipses are of the
accretion disk and suggest that $i$ is large. However, no eclipses are
observed in the X--ray, which implies that the compact object is not
eclipsed by the secondary star, and negates the possibility that
$i\sim90\degr$ %
\cite{OroszBailyn:1997}%
. Nevertheless, Zhang et al.\ %
\citeyear{Zhang_et_al.:1997} %
noted that the high inclination angle of \groj\ is the largest amongst
all well-known Galactic black hole binaries. %

\item
The systematic velocity of \groj, $\gamma =
-142.4\pm1.6\,\mathrm{km\,s^{-1}}$ %
\cite{OroszBailyn:1997}%
, is atypically high for a LMXB containing a black hole. This value is
more suggestive of a neutron star binary. %
Brandt et~al.\ %
\citeyear{Brandt_et_al.:1995}%
\ suggested that the progenitor of the primary star in \groj\ did not
collapse directly into a black hole. Rather, it was formed when a
neutron star evolved into a black hole due to the accretion of
additional mass. This evolutionary process would account for the
abnormal $\gamma$--velocity. Mirabel et~al.\ %
\citeyear{Mirabel_et_al.:2002:unchecked}%
\ confirmed that both the systematic velocity and the high
galactocentric eccentricity ($e=0.34\pm0.05$) of this system could be explained by a
natal explosion. %

\end{itemize}

%%%%%%%%%%%%%%%%%%%%%%%%%%%%% Reasons to Study J1655 %%%%%%%%%%%%%%%%%%%%

\subsection{Reasons to Study \groj}\label{cha:GROJ1655-40:sec:IntroductionToJ1655:subsec:ReasonsToStudyJ1655}

If we can assume a negligible disk contribution, the unique
ellipsoidal variations and eclipses of \groj\ should allow us to constrain
the mass ratio and inclination considerably. This will lead to a unprecedented precision in the measurement of the
mass ratio and orbital inclination -- see, for example, %
\citeN{OroszBailyn:1997}.%
\ The high precision in $q$ and $i$ in return should enable a similarly precise value
for the mass of the primary to be determined, strengthening the
evidence for the black hole nature of this star. %

\vspace{\myparskip}

\groj\ is also important due to its superluminal jet nature. The study of Galactic superluminal X--ray sources is important as it
may lead to a better understanding of \textbf{Active Galactic Nuclei
(AGN)}. These show similar jet geometry to these Galactic sources,
possibly due to comparable conditions of accretion. AGN are thought to
be powered by massive ($\lsim 10^{10}$\,M\sun) black holes and may be the
high-luminosity counterparts to sources such as \groj. %

\vspace{\myparskip}

Finally, further study of the eclipses in \groj\ may lead to information on the
structure of the disk in this system. \groj\ therefore could be the best X--ray binary with which to test
various models of X--ray production and jet formation in the search for
a better understanding of AGN. %

%%%%%%%%%%%%%%%%%%%%%%%%%%%%% The Accretion Disk in \groj %%%%%%%%%%%%%%%%%%%%

\subsection{The Accretion Disk Contribution in \groj}\label{cha:GROJ1655-40:sec:IntroductionToJ1655:subsec:TheAccretionDisk}

When modelling SXTs, it is often assumed that the accretion disk
contributes little light to the overall system flux during
quiescence (see page~%
\pageref{cha:InfraredDataReductionTechniques:sec:InfraredAstronomy:subsubsec:InfraredSpectroscopy}%
). Indeed, the spectral type of the secondary star in \groj\ is earlier than that in most other SXTs, implying that any small
contamination due to an accretion disk would be less significant than
in the other systems %
\cite{BeerPodsiadlowski:2001}%
. However, it is important to verify the above assumption for \groj, if the derived
system parameters are to be believed. %

\vspace{\myparskip}

Bailyn et~al.\ %
\citeyear{BailynJain_et_al.:1998}%
\ note that the disk in \groj\ is redder than the secondary during
quiescence. The unusually long period of this system implies that a
large disk can form, the outer parts of which will be cooler than in
other systems. These characteristics imply that there may
be more flux at the longer wavelengths, such as the infrared. This questions the validity of the
assumption that the quiescent disk contribution is negligible, even though there is no evidence of such a contribution from photometric measurements (see, for example, %
\citeNP{GreeneBailynOrosz:2001}%
). Therefore, the
disk contribution must be shown to be inconsequential using other
methods such as spectral analysis, if we are to accurately determine
the orbital inclination and mass ratio of \groj. %

%%%%%%%%%%%%%%%%%%%%%%%%%%%%% Observations of \groj\ %%%%%%%%%%%%%%%%%%%%

\section{Observations of \groj}\label{cha:GROJ1655-40:sec:ObservationsOfJ1655}

%%%%%%%%%%%%%%%%%%%%%%%%%%%%% The Observatories and the Instruments %%%%%%%%%%%%%%%%%%%%

\subsection{The Observatories and the Instruments}\label{cha:GROJ1655-40:sec:ObservationsOfJ1655:subsec:ObservatoriesAndInstruments}

%%%%%%%%%%%%%%%%%%%%%%%%%%%%% CIRIM %%%%%%%%%%%%%%%%%%%%

\subsubsection{The Cerro Tololo Infrared Imager at CTIO}\label{cha:GROJ1655-40:sec:ObservationsOfJ1655:subsubsec:CIRIM}

The 1995 and 1998 photometric observations of \groj\ were made using the
1.5m Ritchey-Chretien telescope at the \textbf{Cerro Tololo
Interamerican Observatory (CTIO)} located in Chile. The \textbf{Cerro
Tololo Infrared Imager (CIRIM)}%
\footnote{\label{cha:GROJ1655-40:sec:ObservationsOfJ1655:subsubsec:CIRIM:foot:CIRIM}
\url{http://www.ctio.noao.edu/instruments/ir_instruments/cirim/cirim.html}
}%
, a mercury-cadmium-telluride (HgCdTe) $256\times256$ NICMOS 3
array with a variable pixel scale, was the detector used. %

\vspace{\myparskip}

The $J$-- and $K_s$--band extinction coefficients, $k_J$ and $k_{K_{s}}$ (see
page~%
\pageref{cha:InfraredDataReductionTechniques:sec:MagnitudeScale:subsec:AtmosphericExtinction:topic:k}%
), at the observatory were previously measured
to be approximately $0.11$ mag/airmass and $0.064$ mag/airmass,
respectively %
\cite{Curran:2001}.

%%%%%%%%%%%%%%%%%%%%%%%%%%%%% NIRSPEC %%%%%%%%%%%%%%%%%%%%

\subsubsection{The Near Infrared Spectrometer at KECK}\label{cha:GROJ1655-40:sec:ObservationsOfJ1655:subsubsec:NIRSPEC}

Photometric and spectroscopic data for \groj\ were taken in 2000 at
the 10m Keck II telescope at the \textbf{W.M. Keck Observatory (KECK)}
in Mauna Kea, Hawaii.

\vspace{\myparskip}

The instrument selected for the spectroscopy was the \textbf{Near Infrared
Spectrometer} \textbf{(NIRSPEC)}%
\footnote{\label{cha:GROJ1655-40:sec:ObservationsOfJ1655:subsubsec:NIRSPEC:foot:NIRSPEC}
\url{http://www2.keck.hawaii.edu:3636/realpublic/inst/nirspec/nirspec.html}
}, %
a vacuum-cryogenic, high-resolution spectrograph for the Keck II
telescope. %

\vspace{\myparskip}

As part of NIRSPEC, the infrared \textbf{Slit-viewing Camera (S-CAM)}
operates over the wavelength region 0.95--2.5 microns. This HgCdTe
array has $256 \times 256$ pixels and a field of view of $46\arcsec
\times 46\arcsec$ (a pixel scale of $0.183\arcsec$/pixel). This
detector was used to obtain several images of \groj\ for use in
calibration. %

\vspace{\myparskip}

The extinction coefficient in the $K$--band, $k_K$, at the KECK observatory is
known to be $\sim 0.088$ mag/airmass %
\cite{Curran:2001}. %

%%%%%%%%%%%%%%%%%%%%%%%%%%%%% Details of the Observations %%%%%%%%%%%%%%%%%%%%

\subsection{Details of the Observations}\label{cha:GROJ1655-40:sec:ObservationsOfJ1655:subsec:DetailsOfTheObservations}

See Table~\vref{cha:GROJ1655-40:sec:ObservationsOfJ1655:subsec:DetailsOfTheObservations:subsubsec:2000Spectroscopy:tab:JournalOfPhotometricObservations} for an overview of the observations. %

\subsubsection{1995 Photometry}\label{cha:GROJ1655-40:sec:ObservationsOfJ1655:subsec:DetailsOfTheObservations:subsubsec:1995Photometry}

\groj\ was observed using the CIRIM $J$-- and $K_s$--filters on 1995
June 8--13 UT. The f/13.5 focus was selected, giving a pixel scale of 0.65
arcseconds per pixel (with a corresponding field of view of
$166\arcsec\times166\arcsec$) for our observations. The photometric
conditions varied over the run, with seeing typically
$\sim1.5\arcsec$. %

\vspace{\myparskip}

For each observation, the telescope was centered on the target and a grid of images was obtained.  For the 1995 and 1998 CTIO data (see later), normally
nine images were obtained per grid. %

\begin{table}[htb]
\caption{Journal of Photometric Observations of \groj}\label{cha:GROJ1655-40:sec:ObservationsOfJ1655:subsec:DetailsOfTheObservations:subsubsec:2000Spectroscopy:tab:JournalOfPhotometricObservations}

\begin{minipage}{\linewidth}
\renewcommand{\thefootnote}{\thempfootnote}
\DeclareFixedFootnote{\ut}{YYYY MM DD HH:MM:SS} % For fixed footnote


\begin{center}
\begin{tabular}{|l||||c|c|c|c|c|}

\hline
Year & Start\ut\ (UT) & End\ut\ (UT) & Start (MJD) & End (MJD)
\\\hline\hline\hline\hline

1995 & 1995 06 08 02:13:27 & 1995 06 13 02:37:36 & 49876.59267 & 49881.60844 \\\hline
1998 & 1998 05 30 01:46:09 & 1998 06 04 09:26:20 & 50963.57372 & 50968.89329 \\\hline
2000 & 2000 07 24 06:23:52 & 2000 07 24 06:53:07 & 51749.76657 & 51749.78689 \\\hline

\hline
\hline
\hline

Year & Integration Time (s) & Coadds & Images ($J$) & Images ($K$ or $K_s$)
\\\hline\hline\hline\hline

1995 & 15 & 5 & 9 & 225 \\\hline
     & 3  & 5 & -- & 9       \\\hline
1998 & 30 & 4 & 18 & 162 \\\hline
     & 15 & 4 & 63 & -- \\\hline
2000 & 10 & 1 & -- & 19  \\\hline

\hline

\end{tabular}
\end{center}
\end{minipage}
\end{table}

\vspace{\myparskip}

Table~%
\ref{cha:GROJ1655-40:sec:ObservationsOfJ1655:subsec:DetailsOfTheObservations:subsubsec:2000Spectroscopy:tab:JournalOfPhotometricObservations}%
\ summarizes the observations made of \groj\ during 1995 June as well
as the periods in 1998 and 2000. The observations made in 1995 were obtained while the system was in outburst \cite{Harmon:1996}. %

\subsubsection{1998 Photometry}\label{cha:GROJ1655-40:sec:ObservationsOfJ1655:subsec:DetailsOfTheObservations:subsubsec:1998Photometry}

Photometry of \groj\ was obtained in the $J$-- and $K_s$--bands using
the CIRIM detector during 6 nights between 1998 May 30 and 1998 June 4
UT. The variable pixel scale was again set at 0.65 arcseconds per pixel. These observations were made while the system was in a quiescent state. %

\vspace{\myparskip}

The airmass for each observation was in the range 1.2--2.2 for the
$K_s$--band, and 1.1--2.0 for the $J$--band. Seeing during the run was $\sim 1.4\arcsec$. %

\subsubsection{2000 Photometry}\label{cha:GROJ1655-40:sec:ObservationsOfJ1655:subsec:DetailsOfTheObservations:subsubsec:2000Photometry}

On 2000 July 24 UT, images of \groj\ were obtained in the $K$--band
using the NIRSPEC camera.  The fixed pixel scale of 0.183\arcsec/pixel
gave a field of view of $46\arcsec\times46\arcsec$. Of the 19 images,
6 were selected to form a grid. %

\vspace{\myparskip}

The range of airmasses was 1.98--2.02, and the conditions were
excellent, with a seeing of approximately $0.7\arcsec$. %

\subsubsection{2000 Spectroscopy}\label{cha:GROJ1655-40:sec:ObservationsOfJ1655:subsec:DetailsOfTheObservations:subsubsec:2000Spectroscopy}

\groj\ was observed spectroscopically in the $K$--band using the NIRSPEC
spectrograph on 2000 July 24 UT, using a slit of width 0.57\arcsec and
orientated approximately east-west. The resolution of the spectra was
11.2\,\AA. The airmass for the observations ranged from 1.057 to
2.087. %

\vspace{\myparskip}

An A0 type star (\mbox{HD 326320}) was also observed using similar
spectroscopic settings, and several Ne and Ar arc lamp spectra
were obtained in the $K$--band. %

\vspace{\myparskip}

\begin{table}[!htb]
\caption{Journal of Spectroscopic Observations}\label{cha:GROJ1655-40:sec:ObservationsOfJ1655:subsec:DetailsOfTheObservations:subsubsec:2000Spectroscopy:tab:JournalOfSpectroscopicObservations}

\begin{minipage}{\linewidth}
\renewcommand{\thefootnote}{\thempfootnote}

\begin{center}
\begin{tabular}{|c|c|c|c|}

\hline
Start (UT) & End (UT) & Start (MJD) & End (MJD) \\\hline\hline\hline\hline

2000 07 24 06:21:51 & 2000 07 24 07:18:35 & 51749.76517 & 51749.80457 \\\hline

\hline
\hline
\hline

Target    & Integration Time (s) & Coadds & Images ($K$)  \\\hline\hline\hline\hline

\groj\   & 300 & 1 & 2  \\\hline
HD 326320      & 20  & 1 & 2  \\\hline
Neon Arc Lamp  & 5   & 1 & 2   \\\hline
Argon Arc Lamp & 5   & 1 & 2  \\\hline

\hline

\end{tabular}
\end{center}
\end{minipage}
\end{table}

See Table~%
\ref{cha:GROJ1655-40:sec:ObservationsOfJ1655:subsec:DetailsOfTheObservations:subsubsec:2000Spectroscopy:tab:JournalOfSpectroscopicObservations}%
\ for a more detailed description of the spectroscopic observations made. %

%%%%%%%%%%%%%%%%%%%%%%%%%%%%% Obtaining the Light Curve of GRO J1655-40 %%%%%%%%%%%%%%%%%%%%

\section{Obtaining the Light Curve of \groj}\label{cha:GROJ1655-40:sec:ObtainingTheLightCurve}

Next we discuss how we applied the data reduction techniques
explored in \S~\ref{cha:InfraredDataReductionTechniques} to obtain a
light curve for \groj\ showing the intrinsic variability of the system. %

%%%%%%%%%%%%%%%%%%%%%%%%%%%%% End of Chapter %%%%%%%%%%%%%%%%%%%
