%%%%%%%%%%%%%%%%%%%%%%%%%%%%% General Conclusions %%%%%%%%%%%%%%%%%%%%

\chapter{General Conclusions}\label{cha:GeneralConclusions}

The most significant finding in this thesis is that the accretion disk
in \\%
% WHITE SPACE %
\groj\ contributes negligible flux during quiescence to the overall system flux in the $K$--band. The spectrum of \groj\ is similar to that of an
isolated F5--F7 III-IV star. We can therefore justify modelling
the $K$--band quiescent observations of \groj\ by only considering the ellipsoidal
variability of the secondary star, and can rely on the derived values
for the mass ratio and inclination. %

\vspace{\myparskip}

From such modelling of \groj\ during quiescence, we have derived
values for the mass ratio and inclination of this system of $2.5$--$6$
and $64\degr$--$70\degr$, respectively, which are in good agreement with
prior results. As the disk contamination has been shown to be
negligible, the value for the orbital inclination can be taken as a
valid estimate. We can therefore conclude that the primary star in
\groj\ is a black hole candidate, with a mass of $M_X =
6.8\pm0.7$\,M\sun. %

\vspace{\myparskip}

From our attempts to model \groj\ in outburst, we have determined that
the \textit{ELC} model employed is too simplistic for an adequate fit to
the observations of binary system with an asymmetric bright disk. The
code is, however, able to model the eclipsing of the disk, the
presence of which is consistent with the inclination previously
derived. %

\vspace{\myparskip}

We have calculated the apparent magnitude of \groj\ in both the $J$--
and $K_s$--band. We find a 0.4\,mag difference between our $K_s$--band
result and that of %
\citeN{GreeneBailynOrosz:2001}%
. As we have determined that our colour estimate of $J_{0} - K_{0} =
0.3\pm0.1$\,mag is consistent with our spectral type for the
secondary star in \groj, we are in agreement with %
\citeN{BeerPodsiadlowski:2001}%
\ that there is an error in the value of %
\citeN{GreeneBailynOrosz:2001}%
. %

\vspace{\myparskip}

Finally, we have confirmed that the quiescent absolute $J$ and $K$
magnitudes of this system ($0.2\pm0.2$\,mag and $-0.1\pm0.2$\,mag,
respectively) are as expected for a long period soft X-ray
transient. %

\vspace{\myparskip}

We have utilised the first high signal-to-noise $K$--band spectrum of a
black hole X-ray transient system to confirm the absence of disk
contamination in that system. Similar observations of other transient
systems should be pursued in order to verify the mass of the compact
objects already derived from observations of the ellipsoidal
variability of these binaries. Further observations of SXTs in search of black hole candidates would also benefit from the use of \textit{ELC} to model quiescent light curves. %

%%%%%%%%%%%%%%%%%%%%%%%%%%%%% End of Chapter %%%%%%%%%%%%%%%%%%%%
